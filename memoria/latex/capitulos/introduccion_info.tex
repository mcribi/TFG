% !TeX root = ../tfg.tex
% !TeX encoding = utf8

\chapter{Introducción}
En la última década, el avance de la inteligencia artificial ha impulsado el desarrollo de sistemas predictivos en numerosas áreas, con un impacto especialmente relevante en el ámbito médico. Las técnicas de \textit{Machine Learning} y \textit{Deep Learning} han demostrado un notable potencial para extraer patrones complejos y relaciones no evidentes en datos clínicos e imágenes médicas, permitiendo construir modelos capaces de realizar diagnósticos, predecir la progresión de enfermedades y optimizar la toma de decisiones clínicas con mayor precisión y eficiencia que los métodos tradicionales.

En el contexto específico de la predicción de complicaciones en biopsias pulmonares, el aprendizaje profundo se presenta como una herramienta fundamental para analizar grandes volúmenes de datos tridimensionales obtenidos mediante Tomografía Computarizada (TC). Las redes neuronales convolucionales (CNNs) permiten capturar características radiómicas de alta complejidad, como texturas, densidad o estructura tridimensional del parénquima pulmonar, que son imperceptibles para el ojo humano pero pueden estar asociadas con un mayor riesgo de complicaciones durante el procedimiento. De este modo, se busca un enfoque predictivo que ayude a optimizar la selección de pacientes y a reducir los riesgos asociados a la intervención.

Durante las primeras fases de este trabajo se exploró también un enfoque basado en cortes axiales 2D, que consistía en dividir cada volumen en slices individuales para entrenar modelos convencionales de clasificación de imágenes. Sin embargo, esta aproximación planteaba varios problemas importantes. En primer lugar, aumentaba significativamente el desbalanceo de clases, ya que el número de cortes con información relevante difería mucho entre volúmenes de pacientes con y sin complicaciones. Además, surgían dificultades para etiquetar de forma coherente los cortes: en las zonas extremas del volumen muchas slices no contenían información pulmonar útil, lo que inducía al modelo a aprender patrones irrelevantes o incluso engañosos. Finalmente, al fragmentar el volumen se perdía la coherencia tridimensional esencial para describir la estructura anatómica completa y capturar contextos espaciales importantes. Estas limitaciones motivaron el abandono de la estrategia 2D en favor de un enfoque 3D completo, mejor adaptado para representar la información volumétrica y mantener la relación espacial entre cortes.

El desarrollo de un modelo predictivo en este ámbito plantea múltiples retos técnicos que se abordan de forma sistemática en este trabajo. En primer lugar, el preprocesamiento de imágenes incluye la normalización de intensidades mediante ventanas Hounsfield, la segmentación automatizada del pulmón y la homogenización de resoluciones con interpolación trilineal, asegurando datos de entrada consistentes y comparables. Posteriormente, la extracción de características radiómicas se realiza empleando redes convolucionales 3D capaces de analizar la textura, la morfología y los patrones espaciales presentes en los volúmenes, con el fin de capturar información relevante para la predicción de complicaciones.

El entrenamiento y la validación de los modelos se llevan a cabo utilizando arquitecturas avanzadas como ResNet y DenseNet adaptadas al procesamiento 3D, así como estrategias de transferencia de aprendizaje mediante preentrenamiento en tareas similares y ajuste fino (fine-tuning) sobre el conjunto de datos específico. Este enfoque busca mitigar el problema del tamaño reducido de la base de datos disponible. Además, se abordan los desafíos derivados del desbalanceo de clases ya que los datos fueron llegando de forma progresiva y no siempre estuvieron equilibrados, mediante técnicas como el oversampling o la ponderación de las funciones de pérdida para equilibrar la representación de las clases durante el entrenamiento.

Otro aspecto esencial del trabajo es la interpretabilidad y explicabilidad de los modelos, implementando herramientas como Grad-CAM o SHAP que permiten visualizar y comprender en qué regiones de las imágenes se concentra la atención del modelo al tomar decisiones. Esta capacidad de explicación facilita su validación clínica y refuerza la confianza del personal médico en las predicciones generadas. Finalmente, se explora la integración de datos multimodales, combinando información de imagen con variables clínicas tabulares para construir modelos híbridos capaces de aprovechar conjuntamente la información radiológica y la clínica del paciente.

Es importante destacar que este estudio aborda un problema para el cual actualmente no existe apenas literatura ni investigaciones previas que analicen la predicción del riesgo de complicaciones en biopsias pulmonares mediante modelos de inteligencia artificial. Por ello, representa un trabajo novedoso que abre una línea de investigación relevante para la medicina. Además, se trata de un reto particularmente complejo porque ni siquiera está claro de antemano si es un problema abordable mediante IA: la predicción de complicaciones es difícil incluso para los propios médicos, que suelen basarse en su experiencia e intuición pero no pueden anticipar con certeza el riesgo. Buena parte del tiempo de este trabajo se han producido resultados nulos o modelos sin capacidad predictiva real, por lo que se optó a probar un amplio abanico de técnicas, arquitecturas y estrategias de entrenamiento con el objetivo de identificar aproximaciones que pudieran extraer patrones útiles y mejorar progresivamente el rendimiento del modelo.

El objetivo principal de este trabajo es, por tanto, diseñar, implementar y validar un modelo predictivo basado en técnicas de aprendizaje profundo y radiómica que permita anticipar complicaciones asociadas a biopsias pulmonares guiadas por TC. Con ello se pretende no solo identificar a los pacientes con mayor riesgo, sino también contribuir al desarrollo de herramientas clínicas más seguras y eficientes que permitan optimizar los recursos sanitarios y mejorar la planificación de estos procedimientos.

\endinput
%--------------------------------------------------------------------
% FIN DEL CAPÍTULO. 
%--------------------------------------------------------------------
